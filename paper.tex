%--------- proposal.tex --------
%
% @author Jeffery Russell 4-12-20
%
% File with project proposal information
%
%-------------------------------

\documentclass[12pt,
 reprint,
%superscriptaddress,
%groupedaddress,
%unsortedaddress,
%runinaddress,
%frontmatterverbose, 
%preprint,
%preprintnumbers,
nofootinbib,
%nobibnotes,
%bibnotes,
 amsmath,amssymb,
 aps,
%pra,
%prb,
%rmp,
%prstab,
%prstper,
%floatfix,
]{revtex4-2}



% enables text wrapping
\usepackage{url}
\makeatletter
\g@addto@macro{\UrlBreaks}{\UrlOrds}
\makeatother
\rule{\linewidth}{1pt}

\usepackage{graphicx}% Include figure files
\usepackage{dcolumn}% Align table columns on decimal point
\usepackage{bm}% bold math
%\usepackage{hyperref}% add hypertext capabilities
%\usepackage[mathlines]{lineno}% Enable numbering of text and display math
%\linenumbers\relax % Commence numbering lines

%\usepackage[showframe,%Uncomment any one of the following lines to test 
%%scale=0.7, marginratio={1:1, 2:3}, ignoreall,% default settings
%%text={7in,10in},centering,
%%margin=1.5in,
%%total={6.5in,8.75in}, top=1.2in, left=0.9in, includefoot,
%%height=10in,a5paper,hmargin={3cm,0.8in},
%]{geometry}


\usepackage{setspace}
\doublespacing

\begin{document}

\preprint{APS/123-QED}

\title{Analyzing GDPR Sentiment in the United States}
\thanks{Submitted as a PUBL-201 assignment at RIT}%

\author{Jeffery B. Russell}
 \email{jeffery@jrtechs.net, jxr8142@rit.edu}
\affiliation{%
 Fourth Year Computer Science Student at RIT\\
 CUBRC Research Assistant\\
 RITlug President
}%

\date{\today}% It is always \today, today,
             %  but any date may be explicitly specified




\begin{abstract}
Conducting qualitative research is essential in implementing public policy because it enables us to better understand our complex political and social environments.
This research project aims to gain a deeper understanding of American's views on privacy so that we can access what types of GDPR (General Data Protection Regulations) like regulations we should implement in the United States.

We found that although most people said that they would support regulations like the GDPR in the United States, most people added stipulations as to how it got implemented and enforced. 
This work calls upon the need to conduct more qualitative research on privacy regulations so that we can find an ideal set of regulations for the United States. 
Despite the varying opinions on implementations, the consensus that there is currently an issue with privacy regulations illustrates the urgent need for policy change at the federal level. 

\begin{description}
\item[Keywords]
COVID-19, Public Policy, Qualitative Research, Artifact Analysis
\end{description}

\end{abstract}
\maketitle





\section{Background}

The recent COVID-19 incident required us to shift our way of living in order to combat the virus. Schools closed, borders closed, everything came to a halt. As people now reconcile working and learning from home, we are learning about the social impact that it has on people daily. Within the RIT community, this pandemic is affecting certain people disproportionately. Many people are finding it hard to adjust to remote learning due to family and other issues where the sudden exodus from RIT caused a housing panic for other people. I have blogged a bit about my own experiences \footnote{\url{https://jrtechs.net/other/working-remote}} as a way of coping with this new normal. With all these struggles, we need to understand how people are affected by the COVID-19 policies so that we can learn how to mitigate unintended negative externalities better. 

\section{Research Question}

This research topic is going to narrow the scope from the pandemic at large, to specify how the pandemic has affected RIT student's academic lives.

\begin{itemize}
    \item After doing remote learning, what are people's sentiments towards it?
    \item Has the pandemic affected certain disproportionately more than others?
    \item Has remote learning put tension on people's social and family lives?
\end{itemize}

\section{Current Policy}

This work will specifically look at RIT's policy that closed campus and shifted all learning to be remote. This policy was implemented by RIT on March 15th, 2020, and was communicated to all students via multiple channels of communication\footnote{\url{https://www.rit.edu/news/rit-encourages-students-not-return-campus-courses-resume-march-23-through-alternative-modes}}.

\section{Methodology}

This research conducts two different methods of data collection and coding. After all the data is collected, it will be analyzed holistically. Data collection was done between 4/15/2020 and 4/22/2020. 

\subsection{Material Culture Review}

Based on prior work done with analyzing photographs of an event, this work aims to do the same by asks the participants to take the photographs themselves \cite{photographyMaterialCulture}. 
Photographs are going to be collected from participants that they think describes their time social distancing. Gathering photographs is a powerful technique because it allows people to express what they are feeling in a photograph. Many people may post pictures of their desks, empty streets, or family; the point is to capture the moment.
Analyzing photographs such as the ones taken during WWII\cite{wwII} have provided exciting insight, so doing one in live time on the effects of COVID on students would be interesting. 

Photographs were collected on Discord servers with RIT students\footnote{\url{https://discordapp.com/}}. With photographs, we can extrapolate more sentimental data. Photographs help tell the story and narrative as to what is going on. Although we, as researchers, can imagine what other people are going through, seeing pictures would help the researcher conceptualize it more.

\subsection{Interview}

This research conducted five semi-structured remote interviews of RIT students.
Each interview took approximately a half hour to conduct. During each interview field notes were recorded and can be found in Appendix B. 

Applied Action in conjunction with the Critical Humanism framework was used to analyze and learn each person's truth. 
Critical Humanism falls on the radical change and subjective views spectrum. 
We chose this Critical Humanism because we are seeking to pull out varying viewpoints from people and enact change with them.

Appendix A illustrates the template interview script.
Since this is following the action research paradigm, an unstructured interview process allowed us to probe the interviewee better and pull out relevant information.
The interview template contains the important questions being asked, along with probing questions.


\section{Findings}

This section discusses what was discovered by conducting the interviews and collecting images.

\subsection{Interviews}

Field notes of the interviews can be found in Appendix B. After the interviews were conducted, coding and theme analysis was done. 

A lot can be said about each participant individually, however, there were some major themes observed. First was the stark differences in the degree to which the person was affected. 


% where is home


\subsubsection{Home}

For many students our age, the question of where home is not a decisive answer.
Many college students at RIT have had autonomy for the past few years and now are more or less forced to live back with parent. However, not everyone has parents that they can go home too or doing so would be putting them at greater risk of getting the virus. People in their twenties are really caught in a weird period of their lives where they are living with parents, roommates, at school, or with a spouse, where "home" is for quarantine can vary.

One common thread was that the decision to close housing on RIT campus created a great panic for many students as to where they would spend the next few months. Personally for me I had to close an apartment and move half way across the state. Other people interviewed had to move their stuff across the country.
Two of the participants were agitated that RIT decided to close on campus apartments since they were probably just as safe as where they call home. 


\subsubsection{Work}

% "work"


\subsubsection{Online Classes}
% RIT Shift to online classes


\subsubsection{Productivity}
% getting work done remotely


\subsubsection{Socially}
% socially


\subsubsection{Future Policies}
% future policies







\subsection{Photography Artifact Review}

The initial call for photographs received relatively little feedback, however, after following up with people I was able to get more people to submit photographs.

I ended getting some really depressing results as results for photo submissions -- two photographs received had a noose in it. This did not come at a great shock because the general sentiment on a lot of online discourses have been extremely negative over the past few weeks of remote learning. In particular the RIT Reddit \footnote{\url{https://www.reddit.com/r/rit/}} has had a lot of people vent and express their concerns and trials of remote learning. Reddit has traditionally been known for a healthy level of narcissism; however, this last month has been exceptional. It did come at a shock that people sent me those images over Discord because it is not anonymous and I know everyone there. These images were probably sent in a satirical way, however, I am not excluding them from this research because they were the first images received (almost immediately) and represent how dark the mood has gotten recently.

Three codes were used to initially divide the images into further analysis:

\begin{itemize}
    \item Work environment
    \item Socially
    \item Mood
\end{itemize}

\subsubsection{Work Environment}

This category covers all the images that reflected the work environment of remote learning.

\subsubsection{Socially}

This category covers the social aspects of remote learning. 

\subsubsection{Mood}

This category covers everything that covers the mood of remote learning. 

\subsection{Common Themes}

Between both the artifact analysis and interviews a few things can be noted. The first is the vast differences in the respondents and their different experiences as remote learners during this pandemic. Academia is supposed to be the great equalizer but, now that we are all tossed in different environments that is no longer the case. We see that each participant had a vastly different experience with remote learning. While some people had minor difficulties with the adjustment, others saw it as a great hurdle to jump over.


\section{Discussion}

Also it is worth noting that 



\subsection{Future Work}

The first thing worth mentioning is that this research is not all encompassing. When collecting images one alumn noted "not a student, but I sure hope accessibility is a part of your research
especially considering the crowd RIT has". Although I did not include accessibility in my research, that is an area of potential research. Given that I don't know any deaf students, conducting that accessibility research would have been difficult to initialize while social distancing because I am not yet embedded in that community.
I collected images and chose interviews based on people that I knew and who were accessible on various platforms for social interaction like Slack, Zoom and Discord. Acknowledging this, we didn't manage to interview people that were socially isolated and not a part of any friend circles. Future work can entail researching a broader scope of people. 

There are two major areas for future research questions with this realm. The first is how do we better optimize remote learning. We found that many students struggled with the transition to online learning. The second area for future work lies in how can we improve how we self isolate from each other. We are biologically wired to be social animals, so while we may do well in self isolation in the short term, in the longer duration we struggle to stay motivated and remain in a positive mind set. Although certain people find different coping mechanisms to help themselves be better remote learners, there is no one method works with all. A slew of factors ranging from economic, social, and mental all tie into the success of someone working remote. It is important that we look at how we craft public policy because it affects everyone. Remote learning was before optional but with the pandemic it became mandatory.

\section{Acknowledgment}

This was submitted as a RIT PUBL-201 project for professor Blankley's class.
The code used to generate this report can be found on the researchers Github\footnote{\url{https://github.com/jrtechs/PUBL-201-FINAL/}}.

\subsection{\label{appendix:interview}Appendix: Interview Script}


\begin{itemize}
    \item What year student are you at RIT?
    \begin{itemize}
        \item What is your major?
        \item When do you graduate?
    \end{itemize} 
    \item How have you been affected by the decision to close down campus?
    \begin{itemize}
        \item Did you have to find a new apartment? Living with family?
        \item Did you already retrieve your things from campus?
    \end{itemize}
    \item How have remote classes been going?
    \begin{itemize}
        \item Do you feel better or worse about remote learning?
        \item Did the pass/fail option lessen any stress?
        \item Do you feel like your getting enough information out of the class as compared to physical class meetings?
    \end{itemize}
    \item Where do you typically get most of your work done?
    \begin{itemize}
        \item Is it hard to get motivated to get work done?
        \item Has your sleep schedule changed?
            \begin{itemize}
                \item Are you working at weird hours?
                \item Is this a positive or negative thing for you?
            \end{itemize}
        \item Do you have more distractions when working remotely?
        \begin{itemize}
            \item Internet?
            \item Family?
            \item Environment (noisy neighbors, pets, etc)?
        \end{itemize}
    \end{itemize}
    \item How has doing remote classes affected you socially?
    \begin{itemize}
        \item How are your friends communicating?
        \item How are your family communicating?
        \item Are there certain people/groups of people you no-longer talk to?
        \item Has this put any stress on certain relationships?
    \end{itemize}
    \item Acknowledging that we would have to continue social distancing for at least a few more months, are there any policies that would you like to see RIT implement?
    \begin{itemize}
        \item Are teachers that make assignments open at 8:00AM and due at 9:00AM putting an undue burden on the student?
        \item Should RIT do more to help standardize the way classes are ran online or should each class be unique.
        \item Are online exams fair or accurate?
    \end{itemize}
\end{itemize}


\newpage

\bibliographystyle{plain}
\bibliography{ref}

\end{document}
