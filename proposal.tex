%--------- proposal.tex --------
%
% @author Jeffery Russell 4-12-20
%
% File with project proposal information
%
%-------------------------------

% \documentclass[12pt]{extarticle}
\documentclass[12pt]{apa6}
\usepackage[utf8]{inputenc}
\usepackage{cite}


\usepackage[nobiblatex]{xurl}

\usepackage{hyperref}

\title{PUBL-201 Project Proposal: Analyzing the Effects of Remote Learning on RIT Students}

%
\author{Jeffery B. Russell}
\affiliation{%
 Fourth Year Computer Science Student at RIT\\
 CUBRC Research Assistant\\
 RITlug President
}%


\begin{document}

\maketitle

\section{Topic}

The recent COVID-19 incident required us to shift our way of living in order to combat the virus. Schools closed, borders closed, everything came to a halt. As people now reconcile working and learning from home, we are learning about the social impact that it has on people daily. Within the RIT community, this pandemic is affecting certain people disproportionately. Many people are finding it hard to adjust to remote learning due to family and other issues where the sudden exodus from RIT caused a housing panic for other people. I have blogged a bit about my own experiences \footnote{\url{https://jrtechs.net/other/working-remote}} as a way of coping with this new normal. With all these struggles, we need to understand how people are affected by the COVID-19 policies so that we can learn how to mitigate unintended negative externalities better. 

\section{Research Question}

This research topic is going to narrow the scope from the pandemic at large, to specify how the pandemic has affected RIT student's academic lives.

\begin{itemize}
    \item After doing remote learning, what are people's sentiments towards it?
    \item Has the pandemic affected certain disproportionately more than others?
    \item Has remote learning put tension on people's social and family lives?
\end{itemize}

\section{Current Policy}

This work will specifically look at RIT's policy that closed campus and shifted all learning to be remote. This policy was implemented by RIT on March 15th, 2020, and was communicated to all students via multiple channels of communication\footnote{\url{https://www.rit.edu/news/rit-encourages-students-not-return-campus-courses-resume-march-23-through-alternative-modes}}.

\section{Methodology}

This research conducts two different methods of data collection and coding. After all the data is collected, it will be analyzed holistically. 

\subsection{Material Culture Review}

Based on prior work done with analyzing photographs of an event, this work aims to do the same by asks the participants to take the photographs themselves \cite{photographyMaterialCulture}. 
Photographs are going to be collected from participants that they think describes their time social distancing. Gathering photographs is a powerful technique because it allows people to express what they are feeling in a photograph. Many people may post pictures of their desks, empty streets, or family; the point is to capture the moment.
Analyzing photographs such as the ones taken during WWII\cite{wwII} have provided exciting insight, so doing one in live time on the effects of COVID on students would be interesting. 

I am going to start by collecting photos from about ten friends; however, we will expand the call for photographs to the RIT Reddit where hopefully we will get a dozen or so more. With photographs, we can extrapolate more sentimental data. Also, photographs help tell the story and narrative as to what is going on. Although we, as researchers, can imagine what other people are going through, seeing pictures would help the researcher conceptualize it more.

\subsection{Interview}

This research will conduct between five and six semi-structured remote interviews of RIT students. 

We will use Applied Action in conjunction with the Critical Humanism framework to analyze and learn each person's truth. 
Critical Humanism falls on the radical change and subjective views spectrum. 
We chose this Critical Humanism because we are seeking to pull out varying viewpoints from people and enact change with them.

Appendix A illustrates the template interview script.
Since this is following the action research paradigm, an unstructured interview process allowed us to probe the interviewee better and pull out relevant information.
The interview template contains the important questions being asked, along with probing questions.

\section{Acknowledgment}

This was submitted as a RIT PUBL-201 project for professor Blankley's class.

\subsection{\label{appendix:interview}Appendix A: Interview Script}


\begin{itemize}
    \item What year student are you at RIT?
    \begin{itemize}
        \item What is your major?
        \item When do you graduate?
    \end{itemize} 
    \item How have you been affected by the decision to close down campus?
    \begin{itemize}
        \item Did you have to find a new apartment? Living with family?
        \item Did you already retrieve your things from campus?
    \end{itemize}
    \item How have remote classes been going?
    \begin{itemize}
        \item Do you feel better or worse about remote learning?
        \item Did the pass/fail option lessen any stress?
        \item Do you feel like your getting enough information out of the class as compared to physical class meetings?
    \end{itemize}
    \item Where do you typically get most of your work done?
    \begin{itemize}
        \item Is it hard to get motivated to get work done?
        \item Has your sleep schedule changed?
            \begin{itemize}
                \item Are you working at weird hours?
                \item Is this a positive or negative thing for you?
            \end{itemize}
        \item Do you have more distractions when working remotely?
        \begin{itemize}
            \item Internet?
            \item Family?
            \item Environment (noisy neighbors, pets, etc)?
        \end{itemize}
    \end{itemize}
    \item How has doing remote classes affected you socially?
    \begin{itemize}
        \item How are your friends communicating?
        \item How are your family communicating?
        \item Are there certain people/groups of people you no-longer talk to?
        \item Has this put any stress on certain relationships?
    \end{itemize}
    \item Acknowledging that we would have to continue social distancing for at least a few more months, are there any policies that would you like to see RIT implement?
    \begin{itemize}
        \item Are teachers that make assignments open at 8:00AM and due at 9:00AM putting an undue burden on the student?
        \item Should RIT do more to help standardize the way classes are ran online or should each class be unique.
        \item Are online exams fair or accurate?
    \end{itemize}
\end{itemize}

\bibliographystyle{plain}
\bibliography{ref}

\end{document}
